\documentclass[a4paper]{article}

%% Language and font encodings
\usepackage[english]{babel}
\usepackage[utf8x]{inputenc}
\usepackage[T1]{fontenc}
\usepackage{framed}
\usepackage{amssymb}
\usepackage{dsfont}

%% Sets page size and margins
\usepackage[a4paper,top=3cm,bottom=2cm,left=3cm,right=3cm,marginparwidth=1.75cm]{geometry}

%% Useful packages
\usepackage{amsmath, bm}
\usepackage{graphicx}
\usepackage[colorinlistoftodos]{todonotes}
\usepackage[colorlinks=true, allcolors=blue]{hyperref}
\usepackage{braket}
\usepackage{float}
\usepackage{hyperref}


\title{Science Diary}
\author{Patrick Huembeli}

\begin{document}

\section{Xanadu}

\subsection{Gaussian State}

A gaussian state is a state, where the characteristic function is a gaussian.

Every Gaussian state can be from a thermal state $\nu$ via the transformation $\rho =  U \nu U^{\dag}$.

Single mode Gaussian states can be written as $ \rho = D(\alpha) S(\xi) \nu S^{\dag}(\xi) D^{\dag}( \alpha)$. With the displacement and the squeezing operators D and S.

\subsection{Gaussian vs. Fock backend}



\subsection{What is a coherent state?}

A coherent state is a state of the quantized electromagnetic field. It follows very close to a classical trajectory and was originally found to satisfy the correspondance principle (Quantum systems in limit of big number should describe classical physics).

The canonical coherent state $\ket{ \alpha}$ satisfies $a \ket{ \alpha} = \alpha \ket{ \alpha}$. This behaviour is close to the one of classical light, where the loss of a photon does not affect the state.

\subsubsection{The Field Quadratures}

The dimensionless field quadratures $X$ and $P$ are related to the position and momentum operator of a mass on a spring. And the Hamiltonian of a harmonic oscillator can be rewritten as $H \approx X^2 + P^2$. The electromagnetic field fullfills the following conditions: $Re (E) \approx \cos(X)$ and $Im(E) \approx sin(X)$


\end{document}\grid
